\chapter*{Preface}

\minisec{Goals}
Working with digital images is part of modern microscopy and many imaging techniques rely on digital image processing. Storing, handling, processing and quantifying therefore become part of a scientific procedure and basic knowledge is essential to protect results integrity and prevent scientific misconduct. 

\begin{quote}
	\emph{Scientific image analysis is not photo editing.} (\cite{rossner2004})
\end{quote}

In this course, basic knowledge and techniques for digital image processing are covered, focusing on microscopy image analysis and maintaining scientific integrity. 


\minisec{Acknowledgments}
This script is heavily based on great existing resources that already cover most of the content of this course. These are cited, where applicable, as most content is very basic and covered in pretty much every textbook or tutorial. However, two resources deserve specific mentioning:
\begin{enumerate}
	\item The Fiji Website:\\This website contains a lot of information about how to use Fiji, as well as necessary theoretical background. This should be one of the main resources if you want to continue your studies after this course.
	\item The textbook 'Basics of Image Processing and Analysis' by Kota Miura, Centre for Molecular \& Cellular Imaging, EMBL Heidelberg.\\This textbook has been used in courses for many years and this course content has been heavily influenced by the EMBL textbook. If you are interested, check out the more advanced EMBL course materials that cover Macro programming or Plugin development or the EMBL BIAS courses.
\end{enumerate}

\minisec{Why a different textbook?}
In general, the EMBL textbooks cover all aspects and could have been used for this course. However, we wanted to include information on manuscript figure preparation and image data integrity. Also, we adjusted the existing course contents to the duration of our introductory course.

\minisec{ImageJ \& Fiji}
This textbook was written using the Fiji distribution of ImageJ (IJ version 1.52a, Java 1.8.0\_66), Omero (version 5.4) and Cellprofiler (version 3.0.0) . Fiji, ImageJ, Omero, Cellprofiler and plugins thereof are the hard work of their respective developers. Whenever you publish work using these softwares make sure to cite their work (see websites for more information). This not only gives credit to the developers, but also helps them to maintain funding and working on these projects in the future!
